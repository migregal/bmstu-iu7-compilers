\chapter{Выполнение индивидуального задания}

\section{Постановка задачи}

Напишите программу, которая в качестве входа принимает произвольное регулярное выражение, и выполняет следующие преобразования:
\begin{enumerate}
\item Преобразует регулярное выражение непосредственно в ДКА.
\item По ДКА строит эквивалентный ему КА, имеющий наименьшее возможное количество состояний.
\item Моделирует минимальный КА для входной цепочки из терминалов исходной грамматики.
\end{enumerate}

\section{Примеры работы программы}

Регулярное выражение: \texttt{(a|b)*|(a|b)|(ac)*|(a|b|c)*}

\subsection{Консольный интерфейс}

\includelisting{run.txt}{Пример запуска программы. Часть 1}

\clearpage

\includelisting{example-1.txt}{Пример запуска программы. Часть 2}

\subsection{Построенные изображения}

На рисунках \ref{img:tree}--\ref{img:min_dfa} представлена визуализация работы программы.

\includeimage
    {tree}
    {f}
    {h} 
    {0.4\textwidth}
    {Распознанное дерево}

\includeimage
    {dfa}
    {f}
    {h} 
    {0.25\textwidth}
    {Построенный ДКА}

\includeimage
    {min_dfa}
    {f}
    {h} 
    {0.2\textwidth}
    {Минимизированный ДКА}

\section{Контрольные вопросы}

\begin{enumerate}
\item Какие из следующих множеств регулярны? Для тех, которые регулярны, напишите регулярные выражения.
    \begin{enumerate}
    \item Множество цепочек с равным числом нулей и единиц.
    
    Не является регулярным множеством.
    \item Множество цепочек из \texttt{\{0, 1\}*} с четным числом нулей и нечетным числом единиц.
    
    Приложил ДКА отдельно
    \item Множество цепочек из {0, 1}*, длины которых делятся на 3.
    
    \texttt{((0|1)(0|1)(0|1))*}
    \item Множество цепочек из {0, 1}*, не содержащих подцепочки 101.

    \texttt{(100)*0*(1|00|000)*0*}
    \end{enumerate}
\item Найдите праволинейные грамматики для тех множеств из вопроса 1, которые регулярны.
\item Найдите детерминированные и недетерминированные конечные автоматы для тех множеств из вопроса 1,
которые регулярны.
\item Найдите конечный автомат с минимальным числом состояний для языка, определяемого автоматом

\texttt{M = (\{A, B, C, D, E\}, \{0, 1\}, d, A, \{E, F\})},

где функция в задается таблицей

\begin{table}[h]
    \centering
    \small
    \caption{Переходы в автомате}
    \label{tbl:cmp}

    \begin{tabular}{|c|c|c|}
        \hline
        Состояние & \multicolumn{2}{c|}{Вход} \\ \cline{2-3}
        & 0 & 1 \\\hline
        A & B & C\\
        B & E & F\\
        C & A & A\\
        D & F & E\\
        E & D & F\\
        F & D & E\\\hline
    \end{tabular}
\end{table}

\end{enumerate}
